\section{Command Options}
In this section, we list all of the options available to each command and the
defaults.

\subsection{{\tt read}}
Reads in an {\tt Object} named {\tt <object\_name>} from file {\tt
<file\_name>} and inserts it into the {\tt ObjectMap}.  It is invoked by:
\begin{verbatim}
read <object_name> <file_name>
\end{verbatim}

\subsection{{\tt write}}
Writes an {\tt Object} named {\tt <object\_name>} from the {\tt ObjectMap} to 
file {\tt <file\_name>}.  It is invoked by:
\begin{verbatim}
write <object_name> <file_name>
\end{verbatim}

\subsection{{\tt create}}
\begin{verbatim}
create <object_name> <object_type>
\end{verbatim}

\subsection{{\tt expand}}
\begin{verbatim}
expand <3d_cart corr name> <3d_sphr corr name>
\end{verbatim}

\subsection{{\tt unexpand}}
\begin{verbatim}
unexpand <3d_sphr corr name> <3d_cart corr name>
\end{verbatim}

\subsection{{\tt image}}
\begin{verbatim}
image <corr_name> <source_name>
\end{verbatim}

\subsection{{\tt unimage}}
\begin{verbatim}
unimage <source_name> <corr_name>
\end{verbatim}

\subsection{{\tt chi2}}
\begin{verbatim}
chi2 <corr1>  <corr2> 
\end{verbatim}

\subsection{{\tt gaussparam}}
\begin{verbatim}
gaussparam <source_name> 
\end{verbatim}

\subsection{{\tt intbump}}
\begin{verbatim}
intbump <object_name>
\end{verbatim}

\subsection{{\tt powpec}}
\begin{verbatim}
powspec <object_name> 
\end{verbatim}

\subsection{{\tt fit}}
\begin{verbatim}
fit <corr_name>
\end{verbatim}

\subsection{{\tt fixtail}}
\begin{verbatim}
fixtail <corr_name> 
\end{verbatim}

\subsection{{\tt list}}
\begin{verbatim}
list
\end{verbatim}

\subsection{{\tt delete}}
\begin{verbatim}
delete <object_name> 
\end{verbatim}

\subsection{{\tt rename}}
\begin{verbatim}
rename <old_name> <new_name> 
\end{verbatim}

\subsection{{\tt help}}
\begin{verbatim}
help
\end{verbatim}

\subsection{{\tt quit}, {\tt stop}, {\tt exit}}
\begin{verbatim}
stop
\end{verbatim}

\subsection{{\tt import}}
\begin{verbatim}
import <filename> 
\end{verbatim}

\subsection{{\tt cd}}
\begin{verbatim}
cd <directory> 
\end{verbatim}

\subsection{{\tt slicerad}}
Makes a file called {\tt <file\_name>} containing a slice of the object along
the angle $\theta=${\tt theta}, $\phi=${\tt phi} in the Bertsch-Pratt
coordinates.  Here $\theta$ is the angle with respect to the longitudinal axis
(the z-axis) and $\phi$ is the angle with respect to the sidewards axis (the
x-axis).  It is invoked by:
\begin{verbatim}
slicerad <object_name> <file_name> {
    theta = <angle in rad>
    phi = <angle in rad>
}
\end{verbatim}
The output from this command is a file which contains a few lines of header
information and 5 columns of data.  As is, the file can be plotted using the 
package {\tt xmgrace}.

\subsection{{\tt slices}, {\tt sliceo}, {\tt slicel}}

\subsection{{\tt sliceso}, {\tt slicesl}, {\tt sliceol}}

